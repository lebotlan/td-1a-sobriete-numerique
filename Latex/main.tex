\documentclass[a4paper]{article}

\usepackage{sujet}

\begin{document}

\titre{TD 1A Algorithmique}{Initiation aux tableaux}
%%
\begin{center}
  \Large\bf Communication numérique et tempérance  %% ou : retenue, frugalité, sobriété
\end{center}

\noindent\rule{\linewidth}{.6pt}

\section*{Contexte}

Afin de regagner de la souveraineté sur le secteur stratégique des réseaux sociaux pour collégiens, votre société a pour mission de développer une nouvelle application de communication rapide et bienveillante~: Bikbok.
%% et surtout sensibilisant l'utilisateur aux problèmes d'environnement en affichant l'empreinte carbone du message

Votre équipe est chargée d'écrire un démonstrateur capable d'\textbf{envoyer} les messages. Une autre équipe se charge de la réception et de l'affichage. Les types des messages sont :

\begin{itemize}[itemsep=0.2ex]
\item[$\cdot$] Texte
\item[$\cdot$] Photo ?
\item[$\cdot$] Vocal
\item[$\cdot$] Vidéo
\end{itemize}

Pour la démonstration prévue dans deux semaines, le message envoyé sera par exemple :

\smallskip
\centerline{\hword{Pour ce soir (@kollok) : pâtes ou pizza ?}}
\smallskip

Malgré le caractère confidentiel du message, il  est envoyé en clair -- sans être chiffré. Ce message comprend 41 caractères, dure 3 secondes en vocal et en vidéo.

\section{Message texte}

$\star$ Écrire un petit programme qui~:

\begin{itemize}[itemsep=0.2ex]
\item[$\cdot$] Utilise l'acteur Texte ;
\item[$\cdot$] Envoie le message qui sera tapé par l'utilisateur (dans le champ texte de l'application).
\item[$\cdot$] Compte les destinataires apparaissant dans le message (e.g. @Estelle, @Leo, @Thomas). Une fonction auxiliaire sera bienvenue.
\end{itemize}

Une chaîne de caractères (String) est un tableau~:

\begin{center}
  \begin{tabular}{c|cccccccccccccccccccc}
{\large\strut}   Indice & 1 & 2 & 3 & 4 & 5 & 6 & 7 & ... \\[1ex]
{\large\strut}   Cellule & \cell{P} & \cell{o} & \cell{u} & \cell{r} & \cell{\ } & \cell{c} & \cell{e} & ... \\
  \end{tabular}
\end{center}

\lstinputlisting[language=Ada]{Ada/texte.ads}

%%
%% Idée : trouver le destinataire dans le message, sous la forme @blabla   ajouter une fonction est_alphanum
%%

\begin{itemize}[itemsep=0.2ex]
\item[$\cdot$] Coût numérique : afficher le nombre d'octets utilisés
\item[$\cdot$] Évaluer le coût CO2 ? ... ou plutôt par rapport à un SMS
\end{itemize}

\section{Message vocal}

\begin{itemize}
\item[$\star$] Écrire un programme qui enregistre un son et l'envoie tel quel (utiliser l'acteur Vocal ci-dessous).
\item[$\cdot$] Calculer la taille des données envoyées (en octets) pour le message de test, sachant que le son est échantillonné à 44000 Hz et un échantillon occupe 2 octets.
\item[$\cdot$] coût CO2 et comparaison par rapport au message texte ????
\end{itemize}

La musique ou le son se compresse en général très bien. Pour ne pas gaspiller les ressources, cherchons à transmettre moins d'octets.

\begin{itemize}
\item[$\star$] En préambule, écrire le corps de la procédure Copier ci-après. Cette procédure doit copier une partie du tableau Source dans le tableau Dest.
  La partie copiée commence à la position SPos et sa longueur est Len. Elle est copiée à partir de la position DPos du tableau Dest.

  \begin{lstlisting}[language=Ada]
    procedure Copier(Source : T_Tab ; SPos, Dpos, Len : Integer ; Dest : in out T_Tab) 
  \end{lstlisting}
  
  \item[$\star\star$] En utilisant l'acteur Compression, améliorez votre programme pour qu'il transmette moins d'octets.
  \end{itemize}

\lstinputlisting[language=Ada]{Ada/vocal.ads}

\lstinputlisting[language=Ada]{Ada/compression.ads}

\section{Message vidéo}



Format 1920x1080


\hword{highlight box} example

\end{document}
