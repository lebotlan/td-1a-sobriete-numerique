
\emptysection{TD 1A Algorithmique}{Initiation aux tableaux\\Durée : 2H30}
%%
\begin{center}
  \Large\bf Communication numérique et sobriété
\end{center}

\noindent\rule{\linewidth}{.6pt}

\bigskip

Source : les estimations environnementales proviennent de l'Ademe (\url{impactco2.fr}).

\oldsection*{Contexte}

Afin de regagner en souveraineté sur le secteur stratégique des réseaux sociaux pour collégiens, votre société a pour mission de développer une nouvelle application de communication rapide et bienveillante~: Bikbok.
%GA BikBook? 
%% et surtout sensibilisant l'utilisateur aux problèmes d'environnement en affichant l'empreinte carbone du message
\medskip

Votre équipe est chargée d'écrire un démonstrateur capable d'\textbf{envoyer} les messages. Une autre équipe se charge de la réception et de l'affichage. Les types des messages sont :
texte, vocal, ou vidéo.
%GA texte, vocal ou vidéo (pas de virgule entre vocal et ou?)
Pour la démonstration prévue dans deux semaines, le message envoyé sera, par exemple :

\medskip
\centerline{\hword{\ Pour ce soir (@kollok) : pâtes ou pizza ?\ }}
\medskip

Malgré le caractère confidentiel du message, il est envoyé en clair -- sans être chiffré. Ce message comprend 41 caractères, dure 3 secondes en vocal et en vidéo.
%GA j'en compte 42, tu considéres un caractère d'échapement? ou je me gourre
\oldsection{Message texte}

$\star$ Écrire un petit programme qui~:

\begin{itemize}[itemsep=0.2ex]
\item[$\cdot$] Utilise l'acteur Texte ci-après ;
\item[$\cdot$] Récupère le message tapé par l'utilisateur ;
\item[$\cdot$] Compte les destinataires mentionnés dans le message (e.g. @Estelle, @Leo, @Thomas). Une fonction auxiliaire sera bienvenue.
\item[$\cdot$] Envoie le message.
\end{itemize}
%GA j'aurais peut être précisé, un destinataire est précédé du caractère @ (e.g. @Estelle, @Leo, @Thomas)
Une chaîne de caractères (String) est un tableau~:

\begin{center}
  \begin{tabular}{c|cccccccccccccccccccccccccc}
{\large\strut}   Indice & 1 & 2 & 3 & 4 & 5 & 6 & 7 & 8 & 9 & 10 & 11 & 12 & ... \\[1ex]
{\large\strut}   Cellule & \cell{P} & \cell{o} & \cell{u} & \cell{r} & \cell{\ } & \cell{c} & \cell{e} & \cell{\ } & \cell{s} & \cell{o} & \cell{i} & \cell{r} & ... \\
  \end{tabular}
\end{center}

\lstinputlisting[language=Ada]{Ada/texte.ads}

\subsection*{Coût environnemental}

La taille du message se mesure en octets : il faut maximum deux octets par caractère.
Estimons l'ordre de grandeur de l'impact sur le réchauffement climatique (i.e. le coût eCO2, équivalent CO2).
%GA j'aurais mis peut être (i.e. le coût en g - ou kg - eCO2, équivalent CO2)

\begin{itemize}[itemsep=0.2ex]
\item[$\cdot$] Combien de messages envoyez-vous par an ? (à 10.000 près)
\item[$\cdot$] Coût de la transmission : 10~g-eCO2 par Go (grammes équivalent~CO2 par giga-octets).\\
  {\small Source : Ademe, d'après \url{negaoctet.org}}
\item[$\cdot$] Exprimer ce coût dans une unité étudiant-compatible~: l'eq-café (coût~eCO2 équivalent à une tasse de café, en prenant un café de~20cl~: 111~g-eCO2)
\end{itemize}
%GA j'aurais mis peut-être ecafé!
Noter que l'impact principal du numérique provient majoritairement de la \textbf{fabrication} des équipements.
%GA j'aurais mis peut-être le pourcentage que l'on a trouvé pour appuyer le "majoritairement" aux alentours de 80%...
\begin{itemize}[itemsep=0.2ex]
\item[$\cdot$] Convertir le coût eCO2 de fabrication d'un smartphone en équivalent café : 31kg-eCO2.
\item[$\cdot$] À quelle fréquence changez-vous de smartphone ?
\item[$\cdot$] Comparer avec un steak de boeuf : 7kg-eCO2 (oui, kg) {\footnotesize \color{gray}{(à quelle fréquence mangez-vous du boeuf ?)}}
\end{itemize}

\medskip

Note : l'impact environnemental ne se limite pas aux gaz à effet de serre. Nous ne quantifions pas ici l'usage des ressources, notamment l'eau ou les métaux et terres rares largement utilisées pour la fabrication des équipements numériques.
%GA j'aurais mis l'eau, les métaux et les terres rares largement utilisés ...

\oldsection{Message vocal}

\begin{itemize}
\item[$\star$] Écrire un programme qui enregistre un son et l'envoie tel quel (utiliser l'acteur Vocal ci-dessous).
\item[$\cdot$] Calculer la taille des données envoyées (en octets) pour le message de test, sachant que le son est échantillonné à 44000 Hz et un échantillon occupe 2 octets.
\item[$\cdot$] Quel est le rapport de taille entre le message vocal et le message texte ?
\end{itemize}
%GA peut-être : qui enregistre et envoie un son sans compression (utiliser l'acteur ...). 

\lstinputlisting[language=Ada]{Ada/vocal.ads}

La musique ou le son se compresse en général très bien sans altérer la qualité.
Pour ne pas gaspiller les ressources, cherchons à transmettre moins d'octets.

\begin{itemize}
\item[$\star$] En préambule, écrire le corps de la procédure Copier ci-après. Cette procédure doit copier une partie du tableau Source dans le tableau Dest.
  La partie copiée commence à la position SPos et sa longueur est Len. Elle est copiée à partir de la position DPos du tableau Dest.

  \begin{lstlisting}[language=Ada]
    procedure Copier(Source : T_Tab ; SPos, Dpos, Len : Integer ; Dest : in out T_Tab) 
  \end{lstlisting}
  
  \item[$\star\star$] En utilisant l'acteur Compression, améliorez votre programme pour qu'il transmette moins d'octets.
\end{itemize}

\begin{itemize}
\item[$\cdot$] Quel est le rapport de taille entre le message vocal compressé et le message texte ?
\item[$\cdot$] Combien de messages vocaux envoyez-vous par an ?
\item[$\cdot$] Exprimez le coût de ces messages vocaux en eq-café.
\end{itemize}

%GA pour la partie message texte tu avais mis dans la même liste tout à l'infinitif.

\lstinputlisting[language=Ada]{Ada/compression.ads}


\oldsection{Message vidéo}

Pour la vidéo, il convient de distinguer le son et l'image. Le traitement du son est similaire à la question précédente.

Pour la vidéo~:

\begin{itemize}
\item[$\cdot$] Une image d'une vidéo est une matrice de taille 1920x1080. Chaque cellule contient un pixel coloré et occupe 3 octets (rouge, vert, bleu).
\item[$\cdot$] La vidéo est constituée d'une succession d'images, avec un débit de 25 images par seconde (25 fps).
\item[$\cdot$] Calculer la taille de la vidéo non compressée, en octets.
\item[$\cdot$] Quel est le rapport de taille avec le message texte ?
\end{itemize}

La vidéo se compresse très bien en tolérant une baisse de qualité  (facteur de compression de 120).
%GA le facteur c'est 120 ou 1/120?
\begin{itemize}
\item[$\cdot$] Calculer le rapport de taille entre un message vidéo compressé avec le son et un message texte.
\item[$\cdot$] Combien de messages vidéo envoyez-vous par an ?
\item[$\cdot$] Estimer le coût en eq-café par an.
\end{itemize}

