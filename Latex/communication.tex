
\emptysection{TD 1A Algorithmique}{Initiation aux tableaux\\Durée : 2H30}
%%
\begin{center}
  \Large\bf Communication numérique et sobriété
\end{center}

\noindent\rule{\linewidth}{.6pt}

\bigskip

Source : les estimations environnementales proviennent de l'Ademe (\url{impactco2.fr}).

\ifclimasup
\oldsection*{Précisions}
\begin{itemize}
\item Le sujet de TD réel est présenté dans un format latex différent (d'où une mise en page parfois douteuse dans ce document).
\item Il ne s'agit pas d'un sujet de TD sur l'impact environnemental du numérique. C'est principalement un sujet d'algorithmique de 1A, qui est orienté de manière à aborder rapidement quelques questions environnementales.
\item L'espoir est tout de même d'inculquer certains réflexes dans l'usage numérique (envoyer du texte plutôt que de l'audio ou de la vidéo, pour les développeurs : compresser avant de transmettre)
et de signaler que le coût environnemental provient majoritairement de la fabrication du matériel. (Et de mentionner en passant l'impact significatif de l'agriculture / élevage).
\end{itemize}
\fi

\oldsection*{Contexte}

Afin de regagner en souveraineté sur le secteur stratégique des réseaux sociaux pour collégiens, votre société a pour mission de développer une nouvelle application de communication rapide et
de haut niveau intellectuel~: Bikbok.
%GA BikBook? 
%% et surtout sensibilisant l'utilisateur aux problèmes d'environnement en affichant l'empreinte carbone du message
%% [DLB] Oui mais comment faire ? L'impact d'un seul message sera tellement ridicule...
\medskip

Votre équipe est chargée d'écrire un démonstrateur capable d'\textbf{envoyer} les messages. Une autre équipe se charge de la réception et de l'affichage. Les types des messages sont :
texte, vocal, photo, ou vidéo.
Pour la démonstration prévue dans deux semaines, le message envoyé sera, par exemple :

\medskip
\centerline{\hword{\ Pour ce soir (@kollok) : pâtes ou pizza ?\ }}
\medskip

Malgré le caractère confidentiel du message, il est envoyé en clair -- sans être chiffré. Ce message comprend 41 caractères, dure 3 secondes en vocal et en vidéo (sans les silences
de début et de fin).
\oldsection{Message texte}

$\star$ Écrire un petit programme qui~:

\begin{itemize}[itemsep=0.2ex]
\item[$\cdot$] Utilise l'acteur Texte ci-après ;
\item[$\cdot$] Récupère le message tapé par l'utilisateur ;
\item[$\cdot$] Compte les destinataires mentionnés dans le message avec '@' (e.g. @Estelle, @Leo, @Thomas). Une fonction auxiliaire sera bienvenue.
\item[$\cdot$] Envoie le message.
\end{itemize}
Une chaîne de caractères (String) est un tableau~:

\begin{center}
  \begin{tabular}{c|cccccccccccccccccccccccccc}
{\large\strut}   Indice & 1 & 2 & 3 & 4 & 5 & 6 & 7 & 8 & 9 & 10 & 11 & 12 & ... \\[1ex]
{\large\strut}   Cellule & \cell{P} & \cell{o} & \cell{u} & \cell{r} & \cell{\ } & \cell{c} & \cell{e} & \cell{\ } & \cell{s} & \cell{o} & \cell{i} & \cell{r} & ... \\
  \end{tabular}
\end{center}

\lstinputlisting[language=Ada]{Ada/texte.ads}

\begin{reponse}
\lstinputlisting[language=Ada]{Ada/td1a.adb}
\end{reponse}

\subsection*{Coût environnemental}

La taille du message se mesure en octets : il faut maximum deux octets par caractère.
Estimons l'ordre de grandeur de l'impact sur le dérèglement climatique (i.e. le coût en g-eCO2, g équivalent CO2).

\begin{itemize}[itemsep=0.2ex]
\item[$\cdot$] Combien de messages envoyez-vous par an ? (à 10.000 près)
\item[$\cdot$] Coût de la transmission : 10~g-eCO2 par Go (grammes équivalent~CO2 par giga-octets).\\
  {\small Source : Ademe, d'après \url{negaoctet.org}}
\item[$\cdot$] Exprimer le coût annuel dans une unité étudiant-compatible~: l'ecafé (coût~eCO2 équivalent à une tasse de café, en prenant un café de~20cl~: 1 ecafé = 111~g-eCO2)
\end{itemize}

\begin{reponse}
\begin{itemize}
\item Probablement 10.000 ou 20.000 messages par an.
\item En octets : $n \times 41 \times 2$, soit de l'ordre de $820.000$ octets. On peut arrondir à 1 Go, d'où 10g eCO2.
\item En ecafé : 9\% d'un ecafé.
\end{itemize}
\end{reponse}


Noter que l'impact principal du numérique provient majoritairement de la \textbf{fabrication} des équipements (plus de 80\% du coût eCO2 de l'usage numérique).

\begin{itemize}[itemsep=0.2ex]
\item[$\cdot$] Convertir le coût eCO2 de fabrication d'un smartphone en équivalent café : 31kg-eCO2.
\item[$\cdot$] À quelle fréquence changez-vous de smartphone ?
\item[$\cdot$] Comparer avec un steak de boeuf : 7kg-eCO2 (oui, kg) {\footnotesize \color{gray}{(à quelle fréquence mangez-vous de la viande ?)}}
\end{itemize}

\begin{reponse}
\begin{itemize}
\item 1 smartphone = 280 ecafés
\item 1 steak = 63 ecafés
\item Et donc 5 steaks dépassent le coût CO2 d'un smartphone... mais le steak n'a en général pas besoin de métaux rares.
\end{itemize}
\end{reponse}

\medskip

Note : l'impact environnemental ne se limite pas aux gaz à effet de serre. Nous ne quantifions pas ici l'usage des ressources, notamment l'eau, les métaux et terres rares largement utilisés pour la fabrication des équipements numériques.

\oldsection{Message vocal}

\begin{itemize}
\item[$\star$] Écrire un programme qui enregistre et envoie un son sans compression (utiliser l'acteur Vocal ci-dessous).
\item[$\cdot$] Calculer la taille des données envoyées (en octets) pour le message de test, sachant que le son est échantillonné à 44000 Hz et un échantillon occupe 2 octets.
\item[$\cdot$] Quel est le rapport de taille entre le message vocal et le message texte ?
\end{itemize}

\begin{reponse}
\begin{itemize}
\item $44000 \times 2 \times 3$ soit $264.000$ octets (264 ko).
\item Rapport de taille : le son occupe 3000 fois plus de place environ que le texte.
\end{itemize}
\end{reponse}


\lstinputlisting[language=Ada]{Ada/vocal.ads}

La musique ou le son se compresse en général très bien sans altérer la qualité.
Pour ne pas gaspiller les ressources, cherchons à transmettre moins d'octets.

\vfill

\iffalse
\begin{itemize}
\item[$\star$] En préambule, écrire le corps de la procédure Copier ci-après. Cette procédure doit copier une partie du tableau Source dans le tableau Dest.
  La partie copiée commence à la position SPos et sa longueur est Len. Elle est copiée à partir de la position DPos du tableau Dest.

  \begin{lstlisting}[language=Ada]
    procedure Copier(Source : T_Tab ; Spos, Dpos, Len : Integer ; Dest : in out T_Tab) 
  \end{lstlisting}
%%
\vspace{-4ex}
\begin{center}
\begin{tikzpicture}
\small
%%
\newcommand{\caz}[2][]{\node[anchor=north](#1){$\ \cdot \ $} node[inner sep=0,draw=none,above=0]{\vphantom{012y}{\footnotesize #2}};}
\newcommand{\cay}[2][]{\node[anchor=south](#1){$\ \cdot \ $} node[inner sep=0,draw=none,below=0]{\vphantom{012y}{\footnotesize #2}};}
%%
\def\psiz#1{{\tiny #1}}
%
  \matrix [nodes=draw,column sep=2ex] (m)
  {
    \caz{0};& |\caz{1}; & \caz{2}; & \caz{3} & \caz{...} & \caz{...} & \caz[spos]{\psiz{spos}} & \caz[sposp]{\psiz{spos+1}} & \caz[sposl]{...} & \caz{...} & \caz{...} & \caz{...} \\[2ex]
    \cay{0};& |\cay{1}; & \cay{2}; & \cay{...} & \cay[dpos]{\psiz{dpos}} & \cay[dposp]{\psiz{dpos+1}} & \cay[dposl] {...} & \cay {...} & \cay {...} & \cay {...} & \cay {...} & \cay {...} \\
  };
%
\draw[draw,->] (spos.south) ..controls +(0,-1ex) and +(0,1ex) .. (dpos.north) ;
\draw[draw,->] (sposp.south) ..controls +(0,-1ex) and +(0,1ex) .. (dposp.north) ;
\draw[draw,->] (sposl.south) ..controls +(0,-1ex) and +(0,1ex) .. (dposl.north) ;
\end{tikzpicture}
\end{center}
\vspace{-3ex}
%%  
  \item[$\star\star$] En utilisant l'acteur Compression, améliorez votre programme pour qu'il transmette moins d'octets.
\end{itemize}

\begin{reponse}
\lstinputlisting[language=Ada]{Ada/td1b.adb}
\end{reponse}
\fi

\begin{itemize}
\item[$\cdot$] Commençons par retirer les silences au début et à la fin du message vocal.
L'acteur Compression contient deux fonctions pour les détecter.
%
Écrire la fonction Rogner qui renvoie un nouveau tableau, sans les silence de début et de fin.
  \begin{lstlisting}[language=Ada]
   function Rogner(Tab : V.T_Tab) return V.T_Tab is ...
   \end{lstlisting}
%
Bonus (facultatif) : pour éviter un effet de démarrage et de coupure du son désagréable, garder 250ms de silence au début et à la fin, si possible.

\item[$\cdot$] Compresser et envoyer le message vocal rogné.
\end{itemize}



\begin{itemize}
\item[$\cdot$] Quel est le rapport de taille entre le message vocal compressé et le message texte ?
\item[$\cdot$] Combien de messages vocaux envoyez-vous par an ? Exprimer le coût en ecafé par an.
\end{itemize}

\begin{reponse}
\begin{itemize}
\item Le vocal compressé occupe 15 fois moins de place, soit 18ko environ.
\item Rapport avec le message texte : 15 fois moins qu'en non compressé, donc un rapport de 200 environ.
\item Oui, un message vocal est 200 fois plus coûteux qu'un message texte.
\end{itemize}
\end{reponse}


\lstinputlisting[language=Ada]{Ada/compression.ads}


\oldsection{Envoi d'une photo}

\begin{minipage}[t]{0.74\textwidth}
En utilisant l'acteur Photo, écrire un programme qui~:

\begin{itemize}
\item[$\cdot$] Prend une photo,
\item[$\star$] Recadre la photo grâce à une fonction que vous écrivez~:
  \begin{lstlisting}[language=Ada]
    function Recadrer(Img : T_Image) return T_Image is ...
  \end{lstlisting}
  %
  Cette fonction détecte si l'image contient un signe de cadre fait avec les mains. Si oui, elle renvoie l'image située à l'intérieur du cadre, sinon elle renvoie l'image initiale.

\item[$\cdot$] Compresse l'image, envoie l'image
\end{itemize}
\end{minipage}
%%
\hfill
%%
\begin{minipage}[t]{0.25\textwidth}
\includegraphics[width=\linewidth,valign=t]{frame.png}
\end{minipage}
\medskip

\begin{reponse}
\lstinputlisting[language=Ada]{Ada/td1c.adb}
\end{reponse}


La photo est une matrice de taille 1920x1080. Chaque cellule contient un pixel coloré et occupe 3 octets (rouge, vert, bleu).
Le taux de compression d'une image est de l'ordre de 10.

\begin{itemize}
\item[$\cdot$] Quel est le rapport de taille entre la photo compressée et le message texte ?
\item[$\cdot$] Hors sujet : quel est le rapport de taille entre la photo d'un écran avec un programme Ada et le même programme Ada en tant que fichier texte (compter un texte de 2800 octets) ?
\item[$\cdot$] Combien de photos transmettez-vous par an ? (noter que chaque photo prise est probablement envoyée sur votre drive)
\item[$\cdot$] Exprimer le coût de transmission de ces photos en ecafé.
\end{itemize}

\begin{reponse}
\begin{itemize}
\item Photo compressée : $1920 \times 1080 \times 3 / 10$ soit $622$ ko.
\item Rapport avec le message texte : rapport de 8000 environ\\
Une photo coûte 8000 fois plus qu'un petit message texte.
\item Pour un programme Ada : vous transmettez 200 fois plus d'octets avec une photo d'écran qu'avec le fichier source.
\item Combien de photos par an ? Prenons un peu large, par exemple 1000.
\item 1000 photos prennent $0,6$ Go, que l'on arrondit à 1Go -- l'ordre de grandeur est donc une poignée de Go.
\item Coût de transmission (le stockage a un coût négligeable devant la transmission) : 10g eCO2, cf messages texte.
\end{itemize}
\end{reponse}


\lstinputlisting[language=Ada]{Ada/photo.ads}

\oldsection{Message vidéo}

Pour la vidéo, il convient de distinguer le son et l'image. Le traitement du son est similaire à la question~2. Pour les images,
prendre la même taille que les photos (1920x1080) et 25 images par seconde (25 fps).

\begin{itemize}
\item[$\cdot$] Calculer la taille de la vidéo non compressée, en octets. Rapporter à la taille du message texte.
\end{itemize}

La vidéo se compresse très bien en tolérant une baisse de qualité  (facteur de compression de 120 en haute qualité).
%
\begin{itemize}
\item[$\cdot$] Calculer le rapport de taille entre un message vidéo compressé et un message texte.
\item[$\cdot$] Combien de messages vidéo envoyez-vous par an ? Estimer le coût en ecafé par an.
\end{itemize}

\begin{reponse}
\begin{itemize}
\item Video non compressée : $1920 \times 1080 \times 3 \times 25 \times 3$ soit 466 Mo, c'est donc 5.000.000 fois plus que le message texte.
\item Video compressée : 4 Mo (on néglige le son) soit 50.000 fois le message texte.
\item Pour 200 videos de 3s par an, 0,8Go, encore arrondi à 1Go.
\end{itemize}
\end{reponse}


%% TODO : en tant que développeur, faire des mises à jour régulières de l'application : calculer avec 10 millions d'utilisateurs.
